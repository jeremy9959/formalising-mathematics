\hypertarget{cheatsheet-for-tactics}{%
\section{Cheatsheet for tactics}\label{cheatsheet-for-tactics}}

\hypertarget{intro-h}{%
\subsection{\texorpdfstring{\texttt{intro\ h}}{intro h}}\label{intro-h}}

\begin{itemize}
\tightlist
\item
  `Assume h is a proof of P'
\item
  `Assume P is true'
\item
  'Let a be an arbitrary element of X\texttt{in\ a}∀ a ∈ X` term.
\end{itemize}

If the goal is \texttt{P\ →\ \ Q}, creates a hypothesis h : P and
changes the goal to \texttt{Q.}

If the goal is \texttt{∀\ a∈\ X,\ p\ x} then \texttt{intro\ a} creates a
term \texttt{a} of type \texttt{X} and changes the goal to
\texttt{p\ x}.

\hypertarget{intros-h1-h2}{%
\subsection{\texorpdfstring{\texttt{intros\ h1\ h2}}{intros h1 h2}}\label{intros-h1-h2}}

Does several \texttt{intro} commands at once.

\hypertarget{exact-h}{%
\subsection{\texorpdfstring{\texttt{exact\ h}}{exact h}}\label{exact-h}}

\begin{itemize}
\tightlist
\item
  Our hypothesis \texttt{h} establishes the result
\end{itemize}

Applies the hypothesis \texttt{h} to prove the theorem. \texttt{h} must
be exactly the desired goal. .

\hypertarget{assumption}{%
\subsection{\texorpdfstring{\texttt{assumption}}{assumption}}\label{assumption}}

\begin{itemize}
\tightlist
\item
  Now we see that our hypotheses yield the result.
\end{itemize}

If one of the hypotheses exactly proves the theorem, apply that
hypothesis.

\hypertarget{apply}{%
\subsection{\texorpdfstring{\texttt{apply}}{apply}}\label{apply}}

\begin{itemize}
\tightlist
\item
  Given \texttt{h}, to show \texttt{Q} it suffices to show \texttt{P}.
\end{itemize}

If a hypothesis says \texttt{h\ :\ P\ →\ Q} and our goal is \texttt{Q}
then \texttt{apply\ h} replaces \texttt{Q} by \texttt{P}.

\hypertarget{rw-or-rewrite}{%
\subsection{\texorpdfstring{\texttt{rw} or
\texttt{rewrite}}{rw or rewrite}}\label{rw-or-rewrite}}

\begin{itemize}
\item
  Given \texttt{h\ :\ a=b} we can replace \texttt{a} by \texttt{b} (or
  in the goal or in a hypothesis.
\item
  By hypothesis \texttt{h}, we can replace \emph{left side of
  \texttt{h}} with \emph{right side of \texttt{h}}. Or, with the
  \texttt{←} version, we can replace the \emph{right side of \texttt{h}}
  with the \emph{left side of \texttt{h}} in the goal.
\end{itemize}

Given a hypothesis that asserts the equality of two things (\texttt{=}
or \texttt{⟺}), replace one thing by the other.

\hypertarget{to-replace-in-the-goal}{%
\subsubsection{\texorpdfstring{To replace in the
\emph{goal}:}{To replace in the goal:}}\label{to-replace-in-the-goal}}

By default, given \texttt{h\ :\ a=b}, the command \texttt{rw\ h}
replaces \texttt{a} by \texttt{b} The command \texttt{rw\ ←\ h} replaces
\texttt{b} by \texttt{a}.

\hypertarget{to-replace-in-a-hypothesis-use-at.}{%
\subsubsection{\texorpdfstring{To replace in a \emph{hypothesis}, use
\texttt{at}.}{To replace in a hypothesis, use at.}}\label{to-replace-in-a-hypothesis-use-at.}}

Given hypotheses \texttt{hab:\ a=b} and \texttt{hbc:\ b=c} then
\texttt{rw\ hbc\ at\ hab} changes \texttt{hab} to \texttt{a=c}. Note
that you are using \texttt{hbc} to rewrite \texttt{hab}.

\hypertarget{by_contra}{%
\subsection{\texorpdfstring{\texttt{by\_contra}}{by\_contra}}\label{by_contra}}

\begin{itemize}
\tightlist
\item
  We will prove the contrapositive, so assume the conclusion is false.
\end{itemize}

The tactic \texttt{by\_contra} makes the negation of the goal a
hypothesis and changes the goal to \texttt{false}.

\hypertarget{cases}{%
\subsection{\texorpdfstring{\texttt{cases}}{cases}}\label{cases}}

\begin{itemize}
\tightlist
\item
  Given \texttt{P\ ∨\ Q}, we consider separately the cases when
  \texttt{P} is true and when \texttt{Q} is true. (\texttt{P\ ∨\ Q} is a
  hypothesis)
\item
  To prove \texttt{P\ ∧\ Q} we prove \texttt{P} and \texttt{Q}
  separately.
\item
  If we know there exists an \texttt{x} of type \texttt{T} satisfying a
  property \texttt{p\ x}, we can assume separately that \texttt{x} is of
  type \texttt{T} and that \texttt{p\ x} is a hypothesis. In other
  words, decompose a \texttt{∃\ x,\ p\ x} hypothesis into \texttt{∃\ x}
  hypothesis and a \texttt{p\ x} hypothesis.
\end{itemize}

If \texttt{x} is a \emph{hypothesis} about an inductive type then
\texttt{cases} breaks up that hypothesis into its component parts. For
example, if \texttt{h} is a proof of an \texttt{and} term then
\texttt{cases\ h} is are the separate proofs of the terms. If `h' is a
proof about a product, then \texttt{cases\ h} are the separate proofs of
the terms.

\hypertarget{left-and-right}{%
\subsection{\texorpdfstring{\texttt{left} and
\texttt{right}}{left and right}}\label{left-and-right}}

\begin{itemize}
\tightlist
\item
  It suffices to prove the proposition on the left/right.
\end{itemize}

When a goal is made up of two parts, either of which suffices,
\texttt{left} and \texttt{right} replace the goal with the corresponding
part. For example, if the goal is \texttt{P\ ∨\ Q} then \texttt{left} is
like \texttt{apply} for the implication \texttt{P\ →\ P\ ∨\ Q} and right
is \texttt{apply} for \texttt{Q\ →\ P\ ∨\ Q}.

\hypertarget{use}{%
\subsection{\texorpdfstring{\texttt{use}}{use}}\label{use}}

\begin{itemize}
\tightlist
\item
  Then \texttt{x} satisfies the desired conditions.
\end{itemize}

\texttt{use\ x} says to instantiate the \texttt{∃\ y,\ p\ y} clause of a
goal with x, turning the goal into \texttt{p\ x}.

\hypertarget{split}{%
\subsection{\texorpdfstring{\texttt{split}}{split}}\label{split}}

\begin{itemize}
\tightlist
\item
  We consider the component propositions to our conclusion in turn.
\end{itemize}

\texttt{split} breaks up a compound goal (like \texttt{P\ ∧\ Q} or
\texttt{P\ ⟺\ Q}) into subgoals.

\hypertarget{refl}{%
\subsection{\texorpdfstring{\texttt{refl}}{refl}}\label{refl}}

\begin{itemize}
\tightlist
\item
  True by definition.
\end{itemize}
